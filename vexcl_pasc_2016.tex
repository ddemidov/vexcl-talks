\documentclass[@BEAMER_OPTIONS@]{beamer}
    @USE_PGFPAGES@

    \usetheme[alternativetitlepage=true,titleline=true]{Torino}
    \setbeamertemplate{navigation symbols}{}
    \setbeamertemplate{note page}[plain]
    \setbeamertemplate{caption}{\insertcaption}

    \usepackage[utf8]{inputenc}
    \usepackage{graphicx}
    \usepackage{subfigure}
    \usepackage{xspace}
    \usepackage{adjustbox}
    \usepackage{tikz}
    \usepackage{relsize}
    \usepackage{fancyvrb}
    \fvset{fontsize=\footnotesize}
    \RecustomVerbatimEnvironment{verbatim}{Verbatim}{}
    \usepgflibrary{arrows}
    \usetikzlibrary{shadows,decorations.pathreplacing,patterns,shapes}
    \tikzstyle{every picture}=[semithick,>=stealth,remember picture]
    \usepackage{inconsolata}
    \usepackage{listings}
    \lstset{
        language=C++,
        basicstyle=\footnotesize\ttfamily,
        keywordstyle=\color{black}\bfseries,
        commentstyle=\color{chameleon1}\it\rmfamily,
        stringstyle=\color{chameleon1},
        numbers=left,
        numberstyle=\tiny,
        aboveskip=-0.02\baselineskip,
        belowskip=-0.02\baselineskip,
        columns=flexible,
        extendedchars=false,
        showstringspaces=false,
        morekeywords={global,kernel,ulong,size_t,get_global_id,get_global_size}
        }
    \newcommand{\code}[1]{\lstinline|#1|}
    \protected\def\plusplus{{\nolinebreak[4]\hspace{-.05em}\raisebox{.4ex}{\relsize{-3}\bf ++}}\xspace}
    \newcommand{\CXX}{{\rm C}\plusplus}
    \newcommand{\CC}{{\rm C99}\xspace}

    \usepackage{ifthen}
\usetikzlibrary{shadows.blur}
\newlength{\ribbonoffset}
\setlength{\ribbonoffset}{3em}

% corner, color, text
\newcommand{\ribbon}[3]{
  \ifthenelse{\equal{#1}{east}}{%
    \tikzset{ribbonrot/.style={rotate=-45}}
  }{%
    \tikzset{ribbonrot/.style={rotate=45}}
  }
  \begin{tikzpicture}[remember picture, overlay]
    \node[ribbonrot, shift={(0, -\ribbonoffset)}] at (current page.north #1) {
      \begin{tikzpicture}[remember picture, overlay,scale=0.5]
        \node[
            fill=#2,
            text centered,
            minimum width=50em,
            minimum height=1.2em,
            blur shadow,
            shadow yshift=0pt,
            shadow xshift=0pt,
            shadow blur radius=.2em,
            shadow opacity=50,
            text=white
            ](fmogh) at (0pt, 0pt) {%
            \fontfamily{phv}\selectfont\bfseries\tiny#3};
        \draw[
            white,
            dashed,
            line width=.04em,
            dash pattern=on .2em off 1.5\pgflinewidth
            ] (-25em,1em) rectangle (25em,-1em);
      \end{tikzpicture}
    };
  \end{tikzpicture}
}

    \newcommand{\forkme}{\ribbon{east}{chameleon1}{\href{https://github.com/ddemidov/vexcl}{Fork me on GitHub}}}
    \newcommand{\singledevice}{\ribbon{east}{chameleon3}{Single device only}}
    \newcommand{\additive}{\ribbon{east}{chameleon3}{Additive expressions}}

    \tikzset{
        treenode/.style={
            draw,
            fill=white,
            blur shadow,
            shadow xshift=1pt,
            shadow yshift=-1pt,
            shadow blur radius=2pt,
            shadow opacity=40
            }
        }


    \title{VexCL}
    \subtitle{Experiences in developing a C++ wrapper library for OpenCL}

    \author{Denis Demidov}
    \institute{
        Kazan Federal University /\\
        Institute of System Research, Russian Academy of Sciences
        \\ \vspace{\baselineskip}
        }
    \date{08-10.06.2016, PASC16, Lausanne}

\begin{document}

%----------------------------------------------------------------------------
\begin{frame}{}
    \titlepage
\end{frame}

\note{ }

%----------------------------------------------------------------------------
\section{Introduction}

%----------------------------------------------------------------------------
\begin{frame}{VexCL~--- a vector expression template library for OpenCL/CUDA}
    \forkme

    \begin{itemize}
        \item Created for ease of \CXX based GPGPU development:
            \begin{itemize}
                \item Convenient notation for vector expressions
                \item OpenCL/CUDA JIT code generation
                \item Easily combined with existing libraries/code
                \item Header-only
            \end{itemize}
            \vspace{\baselineskip}
        \item Supported backends:
            \begin{itemize}
                \item OpenCL (Khronos \CXX bindings, Boost.Compute)
                \item NVIDIA CUDA
            \end{itemize}
            \vspace{\baselineskip}
        \item The source code is available under MIT license:
            \begin{itemize}
                \item \href{https://github.com/ddemidov/vexcl}{https://github.com/ddemidov/vexcl}
            \end{itemize}
            \vspace{\baselineskip}
    \end{itemize}
\end{frame}

\note[itemize]{
\item VexCL is a vector expression template library for OpenCL. It allows you
    to use convenient matlab-like notation for vector operations and it
    generates the appropriate compute kernels for you automatically.
\item The library is header-only, so you don't have to build it to use it. The
    source code of the library is available on GitHub under very liberal
    MIT license.
}

%----------------------------------------------------------------------------
\begin{frame}[fragile]{Hello OpenCL: vector sum}
    \setbeamercovered{transparent=40}
    \vspace{-1\baselineskip}
    \begin{columns}
        \begin{column}[t]{0.2\textwidth}
            \begin{exampleblock}{OpenCL}
                \begin{adjustbox}{width=0.19\textwidth, height=\textheight, keepaspectratio}
                    \begin{minipage}{\textwidth}
                        \begin{uncoverenv}<1-2>
                            \lstinputlisting[linerange={1-8}]{code/hello-opencl.cpp}
                        \end{uncoverenv}
                        \begin{uncoverenv}<1-2,3>
                            \lstinputlisting[firstnumber=last, linerange={9-35}]{code/hello-opencl.cpp}
                        \end{uncoverenv}
                        \begin{uncoverenv}<1-2,4>
                            \lstinputlisting[firstnumber=last, linerange={36-48}]{code/hello-opencl.cpp}
                        \end{uncoverenv}
                        \begin{uncoverenv}<1-2,5>
                            \lstinputlisting[firstnumber=last, linerange={49-86}]{code/hello-opencl.cpp}
                        \end{uncoverenv}
                        \begin{uncoverenv}<1-2,6>
                            \lstinputlisting[firstnumber=last, linerange={87-90}]{code/hello-opencl.cpp}
                        \end{uncoverenv}
                        \begin{uncoverenv}<1-2>
                            \lstinputlisting[firstnumber=last, linerange={91}]{code/hello-opencl.cpp}
                        \end{uncoverenv}
                    \end{minipage}
                \end{adjustbox}
            \end{exampleblock}
        \end{column}
        \begin{column}[t]{0.7\textwidth}
            \begin{onlyenv}<2->
            \begin{exampleblock}{VexCL}
                \begin{adjustbox}{width=0.83\textwidth, height=\textheight, keepaspectratio}
                    \begin{minipage}{\textwidth}
                        \begin{uncoverenv}<2>
                            \lstinputlisting[linerange={1-5}]{code/hello-vexcl.cpp}
                        \end{uncoverenv}
                        \begin{uncoverenv}<2,3>
                            \lstinputlisting[firstnumber=last, linerange={6-8}]{code/hello-vexcl.cpp}
                        \end{uncoverenv}
                        \begin{uncoverenv}<2,4>
                            \lstinputlisting[firstnumber=last, linerange={9-13}]{code/hello-vexcl.cpp}
                        \end{uncoverenv}
                        \begin{uncoverenv}<2,5>
                            \lstinputlisting[firstnumber=last, linerange={14-16}]{code/hello-vexcl.cpp}
                        \end{uncoverenv}
                        \begin{uncoverenv}<2,6>
                            \lstinputlisting[firstnumber=last, linerange={17-20}]{code/hello-vexcl.cpp}
                        \end{uncoverenv}
                        \begin{uncoverenv}<2>
                            \lstinputlisting[firstnumber=last, linerange={21}]{code/hello-vexcl.cpp}
                        \end{uncoverenv}
                    \end{minipage}
                \end{adjustbox}
            \end{exampleblock}
            \end{onlyenv}
        \end{column}
    \end{columns}
\end{frame}

\note[itemize]{
\item Here is the simplest example of using vexcl: addition of two vectors on a
    gpu card.
\item The first line is the context initialization. We provide a device filter
    to the context constructor and get all compute devices that satisfy the
    filter. Here we filter by type and get all available GPUs.
\item Data allocation and transfer is also simplified. \code{vex::vector}
    constructor allocates memory on device and possibly transfers initial data
    as well. The parameters here are list of command queues and either size or
    input host vector.
\item Line ten does what's needs to be done here. This simple expression leads
    to automatic kernel generation and launch. And then we copy the results
    back to host and see what we got.
}

\end{document}
