\documentclass[@BEAMER_OPTIONS@]{beamer}
    @USE_PGFPAGES@

    \usetheme[alternativetitlepage=true,titleline=true]{Torino}
    \setbeamertemplate{navigation symbols}{}
    \setbeamertemplate{note page}[plain]
    \setbeamertemplate{caption}{\insertcaption}

    \usepackage[utf8]{inputenc}
    \usepackage[russian]{babel}
    \usepackage{graphicx}
    \usepackage{subfigure}
    \usepackage{xspace}
    \usepackage{adjustbox}
    \usepackage{tikz}
    \usepackage{relsize}
    \usepackage{fancyvrb}
    \fvset{fontsize=\footnotesize}
    \RecustomVerbatimEnvironment{verbatim}{Verbatim}{}
    \usepgflibrary{arrows}
    \usetikzlibrary{shadows,decorations.pathreplacing,patterns,shapes}
    \tikzstyle{every picture}=[semithick,>=stealth,remember picture]
    \usepackage{inconsolata}
    \usepackage{listings}
    \lstset{
        language=C++,
        basicstyle=\footnotesize\rmfamily,
        keywordstyle=\color{chameleon1}\bfseries,
        commentstyle=\color{chameleon4}\it\rmfamily,
        stringstyle=\color{chameleon4},
        numbers=left,
        numberstyle=\tiny,
        aboveskip=-0.02\baselineskip,
        belowskip=-0.02\baselineskip,
        columns=flexible,
        extendedchars=false,
        showstringspaces=false,
        morekeywords={global,kernel,ulong,size_t,get_global_id,get_global_size,cl_uint,cl_platform_id}
        }
    \newcommand{\code}[1]{\lstinline|#1|}
    \protected\def\plusplus{{\nolinebreak[4]\hspace{-.05em}\raisebox{.4ex}{\relsize{-3}\bf ++}}\xspace}
    \newcommand{\CXX}{{\rm C}\plusplus}
    \newcommand{\CC}{{\rm C99}\xspace}
    \newcommand{\www}[1]{\href{#1}{#1}}

    \usepackage{ifthen}
\usetikzlibrary{shadows.blur}
\newlength{\ribbonoffset}
\setlength{\ribbonoffset}{3em}

% corner, color, text
\newcommand{\ribbon}[3]{
  \ifthenelse{\equal{#1}{east}}{%
    \tikzset{ribbonrot/.style={rotate=-45}}
  }{%
    \tikzset{ribbonrot/.style={rotate=45}}
  }
  \begin{tikzpicture}[remember picture, overlay]
    \node[ribbonrot, shift={(0, -\ribbonoffset)}] at (current page.north #1) {
      \begin{tikzpicture}[remember picture, overlay,scale=0.5]
        \node[
            fill=#2,
            text centered,
            minimum width=50em,
            minimum height=1.2em,
            blur shadow,
            shadow yshift=0pt,
            shadow xshift=0pt,
            shadow blur radius=.2em,
            shadow opacity=50,
            text=white
            ](fmogh) at (0pt, 0pt) {%
            \fontfamily{phv}\selectfont\bfseries\tiny#3};
        \draw[
            white,
            dashed,
            line width=.04em,
            dash pattern=on .2em off 1.5\pgflinewidth
            ] (-25em,1em) rectangle (25em,-1em);
      \end{tikzpicture}
    };
  \end{tikzpicture}
}

    \newcommand{\forkme}{\ribbon{east}{chameleon1}{\href{https://github.com/ddemidov/vexcl}{Fork me on GitHub}}}
    \newcommand{\singledevice}{\ribbon{east}{chameleon3}{Single device only}}
    \newcommand{\additive}{\ribbon{east}{chameleon3}{Additive expressions}}

    \tikzset{
        treenode/.style={
            draw,
            fill=white,
            blur shadow,
            shadow xshift=1pt,
            shadow yshift=-1pt,
            shadow blur radius=2pt,
            shadow opacity=40
            }
        }


    \title{OpenCL}
    \subtitle{Программная модель, отличия от CUDA}
    \author{Денис Демидов}
    \institute{Институт системных исследований РАН\\
    Казанский Федеральный Университет}
    \date{23.10.2015}


\begin{document}

%----------------------------------------------------------------------------
\begin{frame}{}
    \titlepage
\end{frame}

\note{ }

%----------------------------------------------------------------------------
\section{Введение}

\begin{frame}{Современные GPGPU платформы}
    \begin{columns}
        \begin{column}{0.45\textwidth}
            \begin{block}{CUDA}
                \begin{itemize}
                    \item Проприетарная архитектура
                    \item Работает только на видеокартах NVIDIA
                    \item Высокоуровневый интерфейс
                    \item Ядра (\CXX) компилируются в псевдокод (PTX) вместе с
                        основной программой
                \end{itemize}
            \end{block}
        \end{column}
        \begin{column}{0.45\textwidth}
            \begin{block}{OpenCL}
                \begin{itemize}
                    \item Открытый стандарт
                    \item Поддерживается многими вендорами
                    \item Низкоуровневый интерфейс
                    \item Ядра (\CC) компилируются во время выполнения основной
                        программы
                \end{itemize}
            \end{block}
        \end{column}
    \end{columns}
\end{frame}

\note[itemize]{
\item Today, major GPGPU programming frameworks are NVIDIA CUDA and OpenCL
\item ...
\item The latter distinction allows one to generate an OpenCL kernel tailored
    for the problem at hand.  And that is what I am going to talk about today.
}

\begin{frame}{Поддерживаемые устройства}
    \begin{itemize}
        \item NVIDIA CUDA
            \begin{itemize}
                \item Видеокарты NVIDIA
            \end{itemize}
        \item OpenCL
            \begin{itemize}
                \item Видеокарты NVIDIA, AMD, Intel
                \item Центральные процессоры Intel, AMD, ARM
                \item Мобильные системы, программируемые чипы, \ldots
            \end{itemize}
    \end{itemize}
\end{frame}

\section{Программный интерфейс OpenCL}
\begin{frame}
    \sectionpage
\end{frame}

\begin{frame}{Справочная информация}
    \begin{itemize}
        \item Сайт группы компаний Khronos
            \begin{itemize}
                \item Спецификации: \www{https://www.khronos.org/registry/cl}
                \item Ресурсы: \www{https://www.khronos.org/opencl/resources}
            \end{itemize}
    \end{itemize}
\end{frame}

\begin{frame}{Основные этапы при использовании OpenCL}
    \begin{itemize}
        \item Инициализация
            \begin{itemize}
                \item Выбор платформы
                \item Выбор устройства
                \item Создание контекста
                \item Создание очереди команд
                \item Выделение памяти на устройстве
                \item Компиляция вычислительных ядер
            \end{itemize}
        \item Работа
            \begin{itemize}
                \item Перенос данных
                \item Выполнение расчетов
            \end{itemize}
    \end{itemize}
\end{frame}

\begin{frame}[fragile]{Платформа OpenCL}
    \begin{description}
        \item[Платформа] --- это конкретная реализация OpenCL от производителя
            аппаратного обеспечения (AMD, Intel, Nvidia, и т.д.)
    \end{description}

    Платформа позволяет получить:
    \begin{itemize}
        \item Свойства платформы:
            \begin{itemize}
                \item \code{CL_PLATFORM_NAME}
                \item \code{CL_PLATFORM_VERSION}
                \item \code{CL_PLATFORM_VENDOR}
            \end{itemize}
        \item Список установленных устройств, поддерживаемых платформой
    \end{itemize}
\end{frame}

\begin{frame}[fragile]{С интерфейс}
    \begin{exampleblock}{}
        \begin{lstlisting}
#include <iostream>
#include <vector>
#include <stdexcept>
#include <CL/cl.h>

void check(cl_int return_code) {
    if (return_code != CL_SUCCESS) throw std::runtime_error("OpenCL error");
}
int main() {
    cl_uint np;
    check( clGetPlatformIDs(0, NULL, &np) );
    std::vector<cl_platform_id> platforms(np);
    check( clGetPlatformIDs(np, platforms.data(), &np) );
    char name[256];
    for (auto p : platforms) {
        check( clGetPlatformInfo(p, CL_PLATFORM_NAME, 256, name, NULL) );
        std::cout << name << std::endl;
    }
}
        \end{lstlisting}
    \end{exampleblock}
\end{frame}

\begin{frame}[fragile]{\CXX интерфейс}
    \begin{exampleblock}{}
        \begin{lstlisting}
#include <iostream>
#include <vector>

#define __CL_ENABLE_EXCEPTIONS
#include <CL/cl.hpp>

int main() {
    std::vector<cl::Platform> platforms;
    cl::Platform::get(&platforms);

    for (const auto &p : platforms)
        std::cout << p.getInfo<CL_PLATFORM_NAME>() << std::endl;
}
        \end{lstlisting}
    \end{exampleblock}
    \begin{exampleblock}{}
        \begin{verbatim}
AMD Accelerated Parallel Processing
NVIDIA CUDA
Intel(R) OpenCL
        \end{verbatim}
    \end{exampleblock}
\end{frame}

\begin{frame}[fragile]{Устройство OpenCL}
    \begin{description}
        \item[Устройство] --- конкретное вычислительное устройство,
            поддерживаемое одной из установленных платформ.
    \end{description}
    \begin{exampleblock}{Получение списка устройств для платформы}
        \begin{lstlisting}
std::vector<cl::Device> devices;
p.getDevices(CL_DEVICE_TYPE_ALL, &devices);

for(const auto &d : devices)
    std::cout << "  " << d.getInfo<CL_DEVICE_NAME>() << std::endl;
        \end{lstlisting}
    \end{exampleblock}
    Типы устройств:
    \begin{itemize}
        \item \code{CL_DEVICE_TYPE_ALL}
        \item \code{CL_DEVICE_TYPE_DEFAULT}
        \item \code{CL_DEVICE_TYPE_CPU}
        \item \code{CL_DEVICE_TYPE_GPU}
        \item \code{CL_DEVICE_TYPE_ACCELERATOR}
    \end{itemize}
\end{frame}

\begin{frame}[fragile]{Контекст OpenCL}
    \begin{description}[\;\;]
        \item[Контекст] --- служит для управления объектами и ресурсами
            OpenCL.\\
            С контекстом связаны:
            \begin{itemize}
                \item устройства
                \item программы и ядра
                \item буферы памяти
                \item очереди команд
            \end{itemize}
    \end{description}
    \begin{exampleblock}{Создание контекста}
        \begin{lstlisting}
cl::Context context(devices);
        \end{lstlisting}
    \end{exampleblock}
\end{frame}

\begin{frame}[fragile]{Очередь команд}
    \begin{description}
        \item[Очередь команд] --- позволяет отправить задание на выполнение на
            конкретное устройство.
    \end{description}
    \begin{itemize}
        \item Очередь связана с единственными контекстом и устройством.
        \item Постановка задания в очередь выполняется асинхронно.
        \item Задания выполняются в порядке их постановки в очередь.
            \begin{itemize}
                \item Вычислительные ядра
                \item Операции копирования памяти
            \end{itemize}
    \end{itemize}
    \begin{exampleblock}{Создание очереди}
        \begin{lstlisting}
cl::CommandQueue queue(context, devices[0]);
        \end{lstlisting}
    \end{exampleblock}
\end{frame}

\begin{frame}[fragile]{Выделение и копирование памяти}
    \begin{description}[\;\;]
        \item[Буфер памяти] --- объект, владеющий некоторым объемом памяти в
            контексте.
            \begin{itemize}
                \item Все устройства в контексте могут получить доступ к
                    буферам памяти.
            \end{itemize}
    \end{description}
    \begin{exampleblock}{Выделение и перенос памяти}
        \begin{lstlisting}
std::vector<double> x(1024, 42.0);

cl::Buffer a(context, CL_MEM_READ_WRITE | CL_MEM_COPY_HOST_PTR,
        x.size() * sizeof(x[0]), x.data());

size_t nbytes = 1024 * sizeof(double);
cl::Buffer b(context, CL_MEM_READ_WRITE, nbytes);

queue.enqueueWriteBuffer(b, CL_FALSE, 0, nbytes, x.data());
queue.enqueueReadBuffer(b, CL_TRUE, 0, nbytes, x.data());
        \end{lstlisting}
    \end{exampleblock}
\end{frame}

\begin{frame}[fragile]{Свойства буферов памяти}
    \begin{itemize}
        \item \code{CL_MEM_READ_WRITE}
        \item \code{CL_MEM_WRITE_ONLY}
        \item \code{CL_MEM_READ_ONLY}
            \vspace{\baselineskip}
        \item \code{CL_MEM_USE_HOST_PTR}
        \item \code{CL_MEM_ALLOC_HOST_PTR}
        \item \code{CL_MEM_COPY_HOST_PTR}
            \vspace{\baselineskip}
        \item \code{CL_MEM_HOST_WRITE_ONLY}
        \item \code{CL_MEM_HOST_READ_ONLY}
        \item \code{CL_MEM_HOST_NO_ACCESS}
    \end{itemize}
\end{frame}

\begin{frame}[fragile]{Создание вычислительного ядра}
    \begin{description}[\;\;]
        \item[Ядро] --- функция, исполняющаяся на вычислительном устройстве.
        \item[Программа] --- содержит исходные тесксты и/или скомпилированные
            ядра.
    \end{description}
    \begin{exampleblock}{Создание программы и ядра}
        \begin{lstlisting}
std::string source = R"(
kernel void add(ulong n, global const double *a, global double *b) {
    ulong i = get_global_id(0);
    if (i < n) b[i] += a[i];
}
)";

cl::Program program(context, source);
program.build(devices);

cl::Kernel add(program, "add");
        \end{lstlisting}
    \end{exampleblock}
\end{frame}

\begin{frame}[fragile]{Запуск ядра на выполнение}
    \begin{exampleblock}{Постановка ядра в очередь}
        \begin{lstlisting}
add.setArg(0, static_cast<cl_ulong>(n));
add.setArg(1, a);
add.setArg(2, b);

queue.enqueueNDRangeKernel(add, cl::NullRange, cl::NDRange(n), cl::NullRange);
        \end{lstlisting}
    \end{exampleblock}

    \begin{exampleblock}{Считывание результатов}
        \begin{lstlisting}
queue.enqueueReadBuffer(b, CL_TRUE, 0, nbytes, x.data());
std::cout << x[0] << std::endl;
        \end{lstlisting}
    \end{exampleblock}
\end{frame}

\section{Hello OpenCL}

\begin{frame}{Полный пример}
    \begin{itemize}
        \item Вычислить сумму двух векторов на видеокарте
            \begin{itemize}
                \item \code{A} и \code{B} --- векторы большой размерности
                \item Вычислить поэлементную сумму \code{B = A + B}.
            \end{itemize}
            \vspace{\baselineskip}
        \item Основные шаги
            \begin{enumerate}
                \item Инициализируем контекст
                \item Выделяем память
                \item Переносим входные данные
                \item Проводим вычисления
                \item Забираем результаты
            \end{enumerate}
    \end{itemize}
\end{frame}

\begin{frame}[fragile]{Hello OpenCL: Сумма двух векторов}
    \vspace{-1\baselineskip}
    \begin{columns}
        \begin{column}{0.2\textwidth}
            \begin{minipage}[c][\textheight][c]{\linewidth}
                \begin{exampleblock}{}
                    \begin{adjustbox}{width=0.19\textwidth, height=\textheight, keepaspectratio}
                        \begin{minipage}{\textwidth}
                            \begin{uncoverenv}<1->
                                \lstinputlisting[linerange={1-7}]{code/hello-opencl.cpp}
                            \end{uncoverenv}
                            \begin{uncoverenv}<2->
                                \lstinputlisting[firstnumber=last, linerange={8-33}]{code/hello-opencl.cpp}
                            \end{uncoverenv}
                            \begin{uncoverenv}<3->
                                \lstinputlisting[firstnumber=last, linerange={34-42}]{code/hello-opencl.cpp}
                            \end{uncoverenv}
                            \begin{uncoverenv}<4->
                                \lstinputlisting[firstnumber=last, linerange={43-68}]{code/hello-opencl.cpp}
                            \end{uncoverenv}
                            \begin{uncoverenv}<5->
                                \lstinputlisting[firstnumber=last, linerange={69-72}]{code/hello-opencl.cpp}
                            \end{uncoverenv}
                        \end{minipage}
                    \end{adjustbox}
                \end{exampleblock}
            \end{minipage}
        \end{column}
        \begin{column}{0.7\textwidth}
            \begin{minipage}[c][\textheight][c]{\linewidth}
                \begin{exampleblock}{\small%
                        \only<1>{Разрешаем выбрасывать исключения}%
                        \only<2>{Инициализируем контекст}%
                        \only<3>{Выделяем память}%
                        \only<4>{Компилируем и выполняем ядро}%
                        \only<5>{Забираем результаты}%
                    }
                    \begin{adjustbox}{width=0.7\textwidth, height=\textheight, keepaspectratio}
                        \begin{minipage}{\textwidth}
                            \begin{onlyenv}<1>
                                \lstinputlisting[linerange={1-7}]{code/hello-opencl.cpp}
                            \end{onlyenv}
                            \begin{onlyenv}<2>
                                \lstinputlisting[firstnumber=last, linerange={8-33}]{code/hello-opencl.cpp}
                            \end{onlyenv}
                            \begin{onlyenv}<3>
                                \lstinputlisting[firstnumber=last, linerange={34-42}]{code/hello-opencl.cpp}
                            \end{onlyenv}
                            \begin{onlyenv}<4>
                                \lstinputlisting[firstnumber=last, linerange={43-68}]{code/hello-opencl.cpp}
                            \end{onlyenv}
                            \begin{onlyenv}<5>
                                \lstinputlisting[firstnumber=last, linerange={69-72}]{code/hello-opencl.cpp}
                            \end{onlyenv}
                        \end{minipage}
                    \end{adjustbox}
                \end{exampleblock}
            \end{minipage}
        \end{column}
    \end{columns}
\end{frame}

\section{Терминология: CUDA vs OpenCL}

\begin{frame}
    \sectionpage
\end{frame}

\begin{frame}[fragile]{Иерархия памяти}
    \begin{columns}
        \begin{column}{0.48\textwidth}
            \begin{block}{CUDA}
                \begin{itemize}
                    \item<2-> Глобальная память\\
                        \code{double *p}
                    \item<3-> Разделяемая память\\
                        \code{__shared__ double *p}
                    \item<4-> Константая память\\
                        \code{__constant__ double *p}
                    \item<5-> Локальная память
                \end{itemize}
            \end{block}
        \end{column}
        \begin{column}{0.48\textwidth}
            \begin{block}{OpenCL}
                \begin{itemize}
                    \item<2-> Глобальная память\\
                        \code{global double *p}
                    \item<3-> Локальная память\\
                        \code{local double *p}
                    \item<4-> Константая память\\
                        \code{constant double *p}
                    \item<5-> Приватная память
                \end{itemize}
            \end{block}
        \end{column}
    \end{columns}
\end{frame}

\end{document}
